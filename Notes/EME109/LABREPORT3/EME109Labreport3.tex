\documentclass[11pt]{article}
\usepackage[margin=1in]{geometry}
\usepackage{titlesec}
\usepackage{graphicx}
\usepackage{amsmath}
\graphicspath{{../../Screenshots/}}

\newcommand{\memoheader}[4]{
  \noindent
  \textbf{To:} #1\\
  \textbf{From:} #2\\ 
  \textbf{Date:} #3\\
  \textbf{Subject:} #4\\[1ex]\hrule\vspace{1.5ex}
}

\newcommand{\atmT}{293.15}
\newcommand{\atmP}{100880.52}
\newcommand{\iP}{500000}
\newcommand{\cPcritone}{216200}
\newcommand{\cPcrittwo}{415000}
\newcommand{\cPdestwo}{9300}



\begin{document}
\memoheader{Professor Benjamin Shaw}
            {Miguel Cuaycong}
            {\today}
            {EME 109 LAB 3 Report - Compressible Flow Nozzle Experiment}
\section*{Equations}
$$P_{\text{crit ratio}}=\frac{\text{Absolute chamber } P_c}{\text{Absolute inlet } P_I}$$
$$V_{\text{exit}}=\frac{F}{\dot{m}}$$
$$V_{\text{exit}}=c_p*T_0\bigg[1-\frac{P_e}{P_0}^{\frac{k-1}{k}}\bigg]$$
<<<<<<< HEAD
=======
$$P_b \equiv \text{chamber pressure} $$
$$P_0 \equiv \text{inlet pressure}$$
$$\dot{m} = \rho A V$$
$$\text{speed of sound }C = \sqrt{kRT}$$
$$M_a=\frac{V}{C}$$
$$\frac{T_0}{T}=1+\frac{k-1}{2}M_a^2$$
$$\frac{P_0}{P}=\frac{T_0}{T}^{\frac{k}{k-1}} = (1+\frac{k-1}{2}M_a^2)^{\frac{k}{k-1}}$$
$$\frac{\rho_0}{\rho}=\frac{T_0}{T}^{\frac{1}{k-1}}=(1+\frac{k-1}{2}M_a^2)^{\frac{1}{k-1}}$$
Can assume ideal gas: $$P = \rho RT$$
Choked flow occurs when: $$\frac{P}{P_{0}} = P_{\text{crit ratio}}$$

\begin{center}
$\frac{T^*}{T_0} = 0.833$
\\
$\frac{P^*}{P_0} = 0.528$
\\
$\frac{\rho^*}{\rho_0} = 0.634$

\end{center}
>>>>>>> de3062c8312e0c3798bd131efc5396c3352384da
\section*{Opening}
\section*{Findings/Summary}
\section*{Discussion}
\subsection{Requirements}
\begin{center}
  \begin{tabular}{|l|c|}
    \hline
    table of experimental and theoretical mass flow rate nozzle 1& y/n
    \\
    \hline
    table of experimental and theoretical mass flow rate nozzle 3& y/n
    \\
    \hline
    plot of experimental and theoretical mass flow rate nozzle 1& y/n
    \\
    \hline
    plot of experimental and theoretical mass flow rate nozzle 3& y/n
    \\
    \hline
    flow choked? & y/n
    \\
    \hline
    table of force calibration data & y/n
    \\
    \hline
    plot of force calibration data & y/n
    \\
    \hline
    Hysterisis present & y/n
    \\
    \hline
    polynomial fit if force cal. data using lm() & y/n
    \\
    \hline
    table of the thrust force vs abs chamber pressure nozzle 1  & y/n
    \\
    \hline
    plot of the thrust force vs abs chamber pressure nozzle 1 & y/n
    \\
    \hline
    table of the thrust force vs abs chamber pressure nozzle 3 & y/n
    \\
    \hline
    plot of the thrust force vs abs chamber pressure nozzle 3 & y/n
    \\
    \hline
    table of theoretical thrust force F & --
    \\
    \hline
    a. for the critical pressure ratio & y/n
    \\
    \hline
    b. for the condition where a normal shock is predicted at exit & y/n
    \\
    \hline
    c. design pressure ratio & y/n
    \\
    \hline
    calculate experimental isentropic efficiency of nozzle 1 for crit pressure ratio & y/n
    \\
    \hline
    calculate experimental isentropic efficiency of nozzle 1 for crit pressure ratio & y/n
    \\
    \hline
    calculate experimental isentropic efficiency of nozzle 3 for crit pressure ratio & y/n
    \\
    \hline
    calculate experimental isentropic efficiency of nozzle 3 for crit pressure ratio & y/n
    \\
    \hline
  \end{tabular}
\end{center}
\subsection{Theoretical mass flow rate}
Governing eq: $$\dot{m} = \rho A V$$
The mass flow rate of a converging nozzle can be determined at exit state where:
\\
$\rho = \frac{P_{\text{exit}}}{RT}$
\\
$A = A_{\text{exit}}$
\\
$V = $
\section*{Results}
\section*{Closing}
\newpage
\section{Photos}
\begin{center}
    \includegraphics[width=\textwidth]{lab3devicediag}
    \\
    \includegraphics[width=300pt]{lab3actualdevice.jpg}
\end{center}

\end{document}