\documentclass[11pt]{article}
\usepackage[margin=1in]{geometry}
\usepackage{titlesec}
\usepackage{url}

\newcommand{\memoheader}[4]{
  \noindent
  \textbf{From:} #2\\ 
  \textbf{To:} #1\\
  \textbf{Date:} #3\\
  \textbf{Subject:} #4\\[1ex]\hrule\vspace{1.5ex}
}

\begin{document}
\memoheader{Prof. Gentry}
            {Miguel Cuaycong}
            {\today}
            {EMS 180: Starship Rocket Body (Thin Walled Pressure Vessel Assumption)}
\noindent This document focuses on the manufacturing process selection of a Starship Rocket body.
The body is assumed to be a thin walled pressure vessel made of one type of material which is 301 stainless steel. 
This steel has desireable properties at cryogenic temperatures.
Longitudinal seam welding will be assumed as the method of joining the steel sheets to create a cylindrical thin wall pressure vessel.
\\~\\
Since the part is large and only has a single material, other shaping methods are not suitable. This means composite forming, and additive manufacturing
will not be effective nor necessary. The environmental use case is also important. Since the part is going to space, it needs to be pressurized.
For this use case, joining methods such as fastening, riveting, snap fits are undesirable. An oxidized film is also needed to prevent cold welds in the vacuum of space.
\\~\\
The manufacturing process has four categories (Fig 6.2)[2]. Primary shaping, Secondary processes, Joining, and Finishing.
One process from each category will be selected to create the part.
\\~\\
The material stock for 301 steel is produced by melting in an Arc furnace process. During this process, carbon is controlled to a desired level then Chromium is added to the bath after the steel is deoxidized.
The material stock is now produced ready for Primary shaping process which in this case is Casting the steel to ingots. 
The casting process can adversely affect the properties of liquid steel due to oxidation. 
After the ingots are cast, the material is formed into steel sheets through rolling. During this process, the steel is at an annealed condition. 
Cutting and machining can also be employed to cut the material into desired dimension. 
To finish, a surface treatment process is conducted to remove scale during heat treatment and improve corrosion resistance.
Surface treatment includes multiple bathes in solutions to remove oxidized scales then passivation to produce a stronger oxidized film.[1][3]
\\~\\
301 steel is austenitic which means its microstructure is primarily FCC austenite. 
This microstructure gives the part high strength properties.
However, chromium carbides can form along the austenite grain boundaries which is why great care is needed
when the steel is exposed to elevated temperature(i.e welding). 
These precipitates can adversely affect fracture toughness, ductility, and corrosion resistance. 
Also, cold work can transform austenite to martensite since it is not thermodynamically stable at room temperature.
Martensite is harder than austenite which makes it more brittle. This is also the reason 301 steel should be in an annealed condition during rolling.
[1][3]
\\~\\ 
Process selection of a thin walled pressure vessel for a Starship rocket body can be summarized as follows:
Casting to produce 301 steel and shape to ingots(Primary shaping), Steel sheets are formed through rolling and cutting(Secondary process).
Surface treatment to remove carbon scales and improve oxide film(Finishing), Welding to seal the pressure vessel (Joining).
\newpage
\section*{References}

\noindent[1] H. F. Rush, ``Cryogenic Materials Selection, Availability, and Cost Considerations,'' in \textit{Cryogenic Technology}, NASA Conference Publication CP-2122, NASA Langley Research Center, Hampton, VA, 1982. [Online]. Available: \url{https://ntrs.nasa.gov/citations/19830010491}
\\~\\
\noindent[2] M. F. Ashby, \textit{Materials Selection in Mechanical Design}, 5th ed. Oxford, U.K.: Butterworth-Heinemann (Elsevier), 2017. ISBN: 978-0-08-100599-6.
\\~\\
\noindent[3] Nickel Institute and International Molybdenum Association (IMOA), \textit{Practical Guidelines for the Fabrication of Austenitic Stainless Steels}, 2019. [Online]. Available: \url{https://www.imoa.info/download_files/stainless-steel/Austenitic_Stainless_Steel_Fabrication_Guidelines.pdf}. [Accessed: Nov. 18, 2025]
\end{document}
