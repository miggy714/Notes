\documentclass[11pt]{article}
\usepackage[margin=1in]{geometry}
\usepackage{graphicx}
\usepackage{amsmath}
\usepackage{verbatim}
\graphicspath{{../Screenshots/}}

\begin{document}
LAB AT KEMPER 1120
\tableofcontents
\section{Creep}
Objectives:
\begin{enumerate}
    \item Explain stages of environmental fracture
    \item Explain mechanisms of crack growth in SCC
    \item Explain mechanisms of crack growth in HIC
    \item Explain stages of Creep
    \item Identify creep fracture surface features
    \item Use creep deformation map to Identify the dominant creep mechanism
    \item Predict creep failure time using Larson-Miller Parameter   
\end{enumerate}
\subsection{Stages of environmental fracture}
A slow or stable fracture due to combined action of loads and the environment
in a susceptible material
\begin{enumerate}
    \item Crack nucleus formation
    \begin{itemize}
        \item mechanical strength reduction(Temperature dependent)
        \item formation of microstructural defects or surface micro-cracking
    \end{itemize}
    \item Nucleated crack turns into macroscopic crack
    \begin{itemize}
        \item load opens the crack
        \item corrosion is dominant
    \end{itemize}
    \item Fracture
    \begin{itemize}
        \item not entirely mechanical
        \item can happen at $K<K_{IC}$
    \end{itemize}
\end{enumerate}
\newpage
\subsection{Mechanisms of crack growth in SCC}
Stress corrosion cracking causes:
\begin{center}
    \centering
    \includegraphics[width=\textwidth]{Screenshot 2025-11-24 134628}
\end{center}
\begin{enumerate}
    \item Slip
    \\-External stress causes dislocation and create slip steps 
    \item film rupture
    \\-Slip steps ruptures oxide film at the crack tip
    \item anodic dissolution
    \\-ruptured oxide film causes exposes material to anodic dissolution (corrosion)
    \item repassivation
    \\-new oxide film forms
\end{enumerate}
\subsubsection{Two types of SCC Mechanisms}
\begin{enumerate}
    \item Controlled by Environment:
    \\ - Predominant mechanism is anodic dissolution
    \item Controlled by Stress:
    \\ - Predominant mechanism is brittle fracture
    \\ - Thin film of corrosion products
\end{enumerate}
\subsubsection{Stages of SCC}

\begin{center}
    \centering
    \includegraphics[width=\textwidth]{Screenshot 2025-11-18 105916}
\end{center}
\begin{enumerate}
    \item Formation and rupture of passive layer
    \item crack tip anodic dissolution
    \item Static fracture
\end{enumerate}
\begin{center}
    \includegraphics[width=\textwidth]{Screenshot 2025-11-24 140313.png}
\end{center}
\subsection{Hydrogen induced cracking (HIC)}
\begin{center}
    \includegraphics[width =\textwidth]{Screenshot 2025-11-24 135941.png}
\end{center}
\begin{itemize}
    \item Reported in low and medium carbon steels exposed to sour environments
    \begin{itemize}
        \item Sour environments contain hydrocarbon $H_2S$, carbon dioxide, and water(l).
    \end{itemize}
    \item atomic hydrogen dissolves in material
    \item gashouse hydrogen inside the metal, near the inclusion is formed and creates
    stress concentration at the edge of the trapping creating cracks
\end{itemize}
\begin{center}
    \includegraphics[width=\textwidth]{Screenshot 2025-11-24 140501.png}
\end{center}
\subsection{Stages of creep}
Creep refers to plastic deformation leading to fracture of a material under stress
when subjected to high temperature. $T> (0.3 \text{ to } 0.5) T_{melting}$
\\
Creep leads to loss of tolerance and component/structural failure.
\\
Used in engine components, turbine blades, Boilers, pipelines, etc.
\\~\\
Effect of high temperature:
\begin{itemize}
    \item Reduction of yield and tensile strength
    \item increase in dislocation mobility(diffusion)
    \item Recovery and recrystallization and grain growth.
    \item Dissolution and precipitation of second phases
\end{itemize}

\begin{center}
    \includegraphics[width=\textwidth]{Screenshot 2025-11-24 141440.png}
\end{center}
Stages of creep:
\begin{enumerate}
    \item The strain decreased due to work hardening phenomena associated to dislocation density increase
    \item Strain rate is constant due to balance between hardening and recovery or softening mechanisms
    \item The strain rate increases due to onset of creep damage, characterized by microvoid nucleation and growth mechanisms
\end{enumerate}
\subsubsection{Dislocation-controlled Creep mechanism}
\begin{center}
    \includegraphics[width=\textwidth]{Screenshot 2025-11-24 144653.png}
\end{center}














\newpage
\section{Wear}
\subsection{Three types of wear}
\subsection{Abrasive Wear}
\begin{enumerate}
    \item Occurs when a harder material is pressed against a softer one
    \item 
\end{enumerate}
\subsubsection{Abrasive wear mechanisms}
\begin{enumerate}
    \item Plowing
    \item Cutting
    \item Fracture
\end{enumerate}
\subsubsection{Types of Abrasive Wear}
\begin{enumerate}
    \item two body
    \item three body
    \\-Occurs when theres a particle involved in between two podies
\end{enumerate}
\subsection{Adhesive Wear}
\subsection{Fretting}



\section{Non destructive testing(NDT)}
Evaluation of an object without changing it or altering it in any fashion.
\subsection{Objectives}
\begin{enumerate}
    \item Define NDT and name some NDT techniques
    \item Principles and steps in penetrant testing
    \item Principles of magnetic particle testing
    \item Name advantages and disadvantages of NDT's
\end{enumerate}

\subsubsection{NDT Technique: Visual Inspection}
\begin{center}
    \centering
    \includegraphics[width=\textwidth]{Screenshot 2025-11-18 125439}

\end{center}
\subsubsection{NDT Technique: Penetrant testing (PT)}
Take advantage of capillary action of Penetrant
\begin{itemize}
    \item Good for surface discontinuities
    \item Can be used in production-type environment
    \item Flaws as fine as 1 micron can be detected
    \item Fluorescent dye can be used
    \item Portable
\end{itemize}
Advantages:
\begin{itemize}
    \item Inexpensive
    \item Sensitive
    \item Minimal equipment
    \item Can be applied to irregular shapes
    \item Versatile and Portable
    \item Minimal training required
    \item Ease of interpretation
\end{itemize}
Disadvantages:
\begin{itemize}
    \item 
\end{itemize}

\subsubsection{NDT Technique: Magnetic-Particle Inspection}
Disadvantages:
\begin{itemize}
    \item Crack direction matters
\end{itemize}

\subsubsection{NDT Technique: Eddy-Current inspection}



\subsection{Destructive testing}
Advantages:
\begin{itemize}
    \item Generates useful data for design purposes
    \item Data can be used to establish standards and specifications
    \item Data achieved through ?destructive? testing ususally quantitative.
    \item Service conditions can be measured
    \item Useful life can be predicted
\end{itemize}
Disadvantages:
\begin{itemize}
    \item specimens cant be used after testing
    \item expensive
\end{itemize}
\subsection{}

\section{NDT (II)}
Objectives:
\begin{enumerate}
    \item Principles of ultrasound inspection
    \item principles of radiography inspection
    \item advantages and disadvantage of NDTs
\end{enumerate}

\subsection{Advantages and disadvantages of NDT}

Advantages:
\begin{itemize}
    \item part not changed or altered after examination. (can be reused)
    \item can be examined for conditions internal and at the surface
    \item parts can be examined while in service
    \item many NDT methods are portable 
    \item cost effective
\end{itemize}

Disadvantages:
\begin{itemize}
    \item operator dependent
    \item do not generally provide quantitative data
    \item orientation of discontinuities must be considered
    \item evaluation of some test results are subjective and subject to dispute
    \item some are expensive(radiography)
    \item defined procedures
\end{itemize}
\section{NDT SUMMARY}
Summary of methods:
\begin{itemize}
    \item Visual testing
    \begin{itemize}
        \item Working principles: Direct observation of the surface with or without optical aids to detect visible defects
        \item Detects Surface discontinuity
        \item Advantages: Simple, low cost, quick process.
        \item Limitations: Limited to visible surface defects. Dependence on inspector skill. No quantitative data.
    \end{itemize}
    \item Penetrant testing
    \begin{itemize}
        \item Working principles: Low-viscosity liquid die applied to a clean surface. Seeps through surface cracks and flaws via capillary action. Developer draws out dye.
        \item Surface or internal discontinuity? Surface 
        \item Advantages: Inexpensive, Can be applied to complex shape, High sensitivity to very fine surface cracks
        \item Limitations: Does not work with porous materials, post-cleaning, chemicals can be hazardous 
    \end{itemize}
    \item Magnetic particle testing
    \begin{itemize}
        \item Working principles: Part is magnetized; surface or near-surface discontinuities affect magnetic flux which attracts fine ferromagnetic particles forming an indication
        \item Surface or internal discontinuity? Surface and near-surface
        \item Advantages: Effective in detecting surface and shallow subsurface cracks
        \item Limitations: 
    \end{itemize}
    \item Radiographic testing
    \begin{itemize}
        \item Working principles: Electromagnetic waves(x-ray and gamma rays). Contrast comes from absorption. irregularties or discontinuities appear dark because of scattering
        \item Surface or internal discontinuity?
        \item Advantages:
        \item Limitations: 
    \end{itemize}
    \item Ultrasonic testing
    \begin{itemize}
        \item Working principles: transducer-longitudinal wave
        \item Surface or internal discontinuity? penetrate internal discontinuity
        \item Advantages:
        \item Limitations: Frequency determines depth. 
    \end{itemize}
\end{itemize}

\subsection{Contrast vs Sharpness}
Contras
\begin{verbatim}
   1. Visual Testing (VT)

Orientation effects

- VT relies on reflected light reaching the eye.

- Cracks that are perpendicular to the viewing/lighting direction cast small shadows or specular highlights → easier to see.

- Cracks parallel to the line of sight (very tight, in the same direction as brushing or machining marks) may “blend in” and be missed.

- Inspectors often change viewing angle and use oblique lighting (flashlight at a low angle) to make narrow surface-breaking defects stand out.

Why the limitations?

- Surface only: visible light cannot penetrate opaque solids, so internal defects do not affect surface appearance.

- Human-eye resolution: the eye typically cannot reliably resolve features below ~0.1–0.2 mm without magnification; very fine cracks will not be seen.

- Access and line of sight: if the surface cannot be seen directly (internal bores, complex weld roots), VT cannot be applied or requires aids (mirrors, borescopes).

- Operator dependence: fatigue, eyesight, experience, and lighting strongly influence detection probability.


---

2. Penetrant Testing (PT)

Orientation effects

- PT detects any surface-breaking discontinuity, largely independent of crack direction, as long as there is an open path to the surface.

- Very tight, closed, or oxidized cracks limit capillary penetration → weak or no indication.

- Gravity can slightly affect performance (penetrant may drain out of upward-facing openings), so parts may be oriented to keep penetrant in the flaws during dwell.

Why the limitations? (e.g., porous materials)

- Doesn’t work on porous/rough materials:
	- Penetrant is drawn by capillary action into every small pore and surface cavity, not just cracks.

	- During development, dye seeps back out from everywhere, creating a uniformly stained background. True defect indications are masked → no contrast, no interpretation.


- Surface-breaking flaws only:
	- Penetrant must physically enter the defect; subsurface flaws have no open path to the surface, so there is no way for penetrant to reach them.


- Requires clean, non-contaminated surfaces:
	- Oil, paint, scale, or dirt can block crack mouths, preventing penetrant ingress.

	- Residual penetrant left on the surface after removal will bleed out everywhere, creating false or smeared indications.


- Post-cleaning and environmental issues:
	- Chemicals (solvents, emulsifiers, developers) must be removed to avoid corrosion/contamination and to meet environmental and safety regulations.



---

3. Magnetic Particle Testing (MT)

Orientation effects

- Sensitivity depends strongly on the angle between the magnetic field lines and the discontinuity.

- A flaw is detected when it interrupts or distorts the magnetic flux, creating leakage fields that attract particles.

- Maximum indication occurs when the discontinuity is approximately perpendicular to the magnetic field direction.

- If a crack is parallel to the field, flux lines bypass it with minimal leakage → the crack may be invisible.

- In practice, parts are often magnetized in two or more directions (e.g., longitudinal and circular) to cover flaws of different orientations.

Why the limitations?

- Only ferromagnetic materials:
	- MT relies on a strong difference between the high permeability of the component and the surrounding air.

	- Non-ferromagnetic metals (aluminum, austenitic stainless, copper alloys) do not concentrate flux, so leakage fields are too weak to pull particles and form sharp indications.


- Limited depth of detection:
	- Magnetic leakage fields decay rapidly with depth; deep internal flaws do not produce strong surface leakage, so they are not detected.


- Geometry and accessibility:
	- Sharp corners, varying thicknesses, and holes distort the magnetic field and create non-relevant indications (flux crowding) that can obscure real flaws.


- Residual magnetization:
	- After testing, components can retain magnetism, which may cause problems (e.g., attracting debris, affecting machining or instruments), so demagnetization is often required.



---

4. Radiographic Testing (RT)

Orientation effects

- RT records material thickness and density along the X‑ray/gamma‑ray beam path.

- Volumetric defects (voids, porosity, inclusions) show clearly due to path length or density differences.

- Planar defects (cracks, lack of fusion) are best detected when the crack plane is nearly perpendicular to the beam, so the defect removes noticeable thickness from the beam path.

- If a crack is parallel or almost parallel to the beam, the beam passes along the crack with almost no extra attenuation or thickness change → the crack can be essentially invisible.

- To mitigate this, inspectors may use multiple exposure angles.

Why the limitations?

- Radiation safety:
	- X‑rays and gamma rays are ionizing; exposure can cause serious health effects. Extensive shielding, exclusion zones, and strict procedures are required, raising cost and limiting where RT can be used.


- Need access to both sides (conventional RT):
	- Source on one side and detector/film on the other → thick or inaccessible structures (e.g., buried pipelines, large castings) can be difficult to set up.


- Poor sensitivity to very tight, planar cracks:
	- If a crack does not produce a significant change in either thickness or density along the beam, film/detector contrast is minimal, so detection probability is low.


- Thickness and material limitations:
	- Very thick or dense materials require high-energy sources to penetrate; this reduces image contrast and sharpness and increases exposure times.


- Relatively slow and costly:
	- Requires careful setup, exposure, film processing or digital handling, and interpretation. Equipment, sources, and controls are expensive.



---

5. Ultrasonic Testing (UT)

Orientation effects

- UT uses specular reflection of sound.

- Planar defects oriented perpendicular to the sound beam act like mirrors and reflect a strong echo back to the probe (in pulse‑echo).

- If the defect plane is tilted, much of the sound is reflected away from the transducer → weaker or missing signal.

- Angle-beam probes are used to steer the beam so that it strikes weld flaws and laminations at a favorable angle.

- Rough, branched, or irregular cracks scatter sound, producing diffuse echoes that can be harder to interpret.

Why the limitations?

- Couplant requirement:
	- At an air–solid interface, acoustic impedance mismatch is huge; almost all incident energy is reflected at the surface.

	- A liquid or gel couplant is needed to “bridge” the gap; without it, sound cannot effectively enter the part.


- Surface condition and geometry:
	- Rough or curved surfaces scatter and deflect the beam, reducing signal-to-noise ratio and making coupling inconsistent.


- Near-surface “dead zone”:
	- After a transmit pulse, the transducer rings for a short time; echoes from very close to the surface occur while the probe is still ringing and can be masked.


- Material structure:
	- Coarse-grained or highly attenuative materials (e.g., some cast steels, austenitic welds) cause strong scattering and absorption, raising background noise and limiting useful depth.


- Operator skill and interpretation:
	- Raw A‑scan signals require experience to distinguish between backwall echoes, geometry reflections, and real flaws. Incorrect calibration or misinterpretation can lead to false calls.



---

6. Eddy Current Testing (ECT)

Orientation effects

- Eddy currents flow parallel to the surface in loops beneath and around the probe.

- Defects are best detected when they interrupt many current paths, typically when the crack is perpendicular to the dominant current flow.

- If a crack runs parallel to the current lines, only a small fraction of the currents is disturbed → a smaller impedance change and weaker signal.

- Different coil geometries and scanning directions are used to vary the current flow direction relative to expected flaw orientations.

Why the limitations?

- Only electrically conductive materials:
	- Eddy currents are induced by changing magnetic fields in conductors. Nonconductive materials cannot support such currents, so no measurable impedance change occurs.


- Limited penetration depth (skin effect):
	- At typical test frequencies, current density decays exponentially with depth (skin depth).

	- Higher frequencies → stronger signals but shallower penetration (good for surface cracks).

	- Lower frequencies → deeper penetration but reduced sensitivity and more susceptibility to noise. Deep internal flaws are therefore hard to detect.


- Sensitivity to lift‑off, geometry, and material variations:
	- Small changes in probe distance (“lift‑off”), edges, thickness variations, and changes in conductivity or permeability can produce signals similar in magnitude to defect signals.

	- This makes interpretation more complex and can require reference standards and sophisticated instruments.


- Not ideal for very rough or highly curved surfaces:
	- Irregular lift‑off and changing geometry cause rapidly varying background signals, masking true discontinuities. 
\end{verbatim}






\newpage
\section{Slide 1}
\begin{enumerate}
    \item Stages of environmental fractures
    \item Creep
    \item Third item
\end{enumerate}

\section{Slide 2 - Environmental Fracture}
slow or stable fracture due to combined action of loads and the environmentin a susceptible material
\subsection{example: Rolled up plastic hoses}
\section{Slide 3 - Stages of Environmental Fracture}
\begin{itemize}
    \item Stage 1: Accumulation of internal damage
    \item Stage 2: Nucleated crack turns into a macroscopic crack
    \item Stage 3: Rapid fracture, can happen $K<K_{IC}$
\end{itemize}
\section{Slide 4 - Creep}
\section{Stress corrosion cracking (SCC) Mechanism}
\begin{itemize}
    \item Slip-dissolution mechanism involves 4 stages: slip, film rupture, anodic dissolution, repassivation
    \item External stress causes dislocations to move and create slip steps, increasing plastic strain, and breaking the oxide film at the crak tip.
\end{itemize}
\section{SCC Mechanisms}
\subsection{Stress corrosion cracking SCC}
For stage I: $log(da/dt) = cK^m$\\
For stage II: $log(da/dt) = M/(zF\lambda)i_{curr}$
\subsection{SCC - Effect of Microstructure}
\subsection{Example problem}
$PLM = (500+273)[17 + log(28*365*24)]$
\\$PLM = 17190$\\
using the graph, $\sigma = 400 MPa$ 


\end{document}

