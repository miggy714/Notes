\documentclass{article}
\title{EMS 182 Notes}
\date{\today}
\begin{document}
\maketitle
\section{Slide 1}
\begin{enumerate}
    \item Stages of environmental fractures
    \item Creep
    \item Third item
\end{enumerate}
\section{Slide 2 - Environmental Fracture}
slow or stable fracture due to combined action of loads and the environmentin a susceptible material
\subsection{example: Rolled up plastic hoses}
\section{Slide 3 - Stages of Environmental Fracture}
\begin{itemize}
    \item Stage 1: Accumulation of internal damage
    \item Stage 2: Nucleated crack turns into a macroscopic crack
    \item Stage 3: Rapid fracture
\end{itemize}
\section{Slide 4 - Creep}
\section{Stress corrosion cracking (SCC) Mechanism}
\begin{itemize}
    \item Slip-dissolution mechanism involves 4 stages: slip, film rupture, anodic dissolution, repassivation
    \item External stress causes dislocations to move and create slip steps, increasing plastic strain, and breaking the oxide film at the crak tip.
\end{itemize}
\section{SCC Mechanisms}
\subsection{Stress corrosion cracking SCC}
For stage I: $log(da/dt) = cK^m$\\
For stage II: $log(da/dt) = M/(zF\lambda)i_{curr}$
\subsection{SCC - Effect of Microstructure}
\subsection{Example problem}
$PLM = (500+273)[17 + log(28*365*24)]$
\\$PLM = 17190$\\
using the graph, $\sigma = 400 MPa$ 


\end{document}

