\documentclass[11pt]{article}
\usepackage[margin=1in]{geometry}
\usepackage{graphicx}
\graphicspath{{../Screenshots/}}

\begin{document}
LAB AT KEMPER 1120
\section{Objectives}
\begin{enumerate}
    \item Explain stages of environmental fracture
    \item Explain mechanisms of crack growth in SCC
    \item Explain mechanisms of crack growth in HIC
    \item Explain stages of Creep
    \item Identify creep fracture surface features
    \item Use creep deformation map to Identify the dominant creep mechanism
    \item Predict creep failure time using Larson-Miller Parameter   
\end{enumerate}
\subsection{Stages of environmental fracture}
A slow or stable fracture due to combined action of loads and the environment
in a susceptible material
\begin{enumerate}
    \item Crack nucleus formation
    \begin{itemize}
        \item mechanical strength reduction(Temperature dependent)
        \item formation of microstructural defects or surface micro-cracking
    \end{itemize}
    \item Nucleated crack turns into macroscopic crack
    \begin{itemize}
        \item load opens the crack
        \item corrosion is dominant
    \end{itemize}
    \item Fracture
    \begin{itemize}
        \item not entirely mechanical
        \item can happen at $K<K_{IC}$
    \end{itemize}
\end{enumerate}
\subsection{Mechanisms of crack growth in SCC}
Stress corrosion cracking causes:
\begin{enumerate}
    \item Slip
    \\-External stress causes dislocation and create slip steps 
    \item film rupture
    \\-Slip steps ruptures oxide film at the crack tip
    \item anodic dissolution
    \\-ruptured oxide film causes exposes material to anodic dissolution (corrosion)
    \item repassivation
    \\-new oxide film forms
\end{enumerate}
\subsubsection{Two types of SCC Mechanisms}
\begin{enumerate}
    \item Controlled by Environment:
    \\ - 
    \item Controlled by Stress:
    \\ - 
\end{enumerate}
\subsubsection{Stages of SCC}

\begin{center}
    \centering
    \includegraphics[width=\textwidth]{Screenshot 2025-11-18 105916}

\end{center}




\section{Wear}
\subsection{Three types of wear}
\subsection{Abrasive Wear}
\begin{enumerate}
    \item Occurs when a harder material is pressed against a softer one
    \item 
\end{enumerate}
\subsubsection{Abrasive wear mechanisms}
\begin{enumerate}
    \item Plowing
    \item Cutting
    \item Fracture
\end{enumerate}
\subsubsection{Types of Abrasive Wear}
\begin{enumerate}
    \item two body
    \item three body
    \\-Occurs when theres a particle involved in between two podies
\end{enumerate}
\subsection{Adhesive Wear}
\subsection{Fretting}



\section{Non destructive testing(NDT)}
Evaluation of an object without changing it or altering it in any fashion.
\subsection{Objectives}
\begin{enumerate}
    \item Define NDT and name some NDT techniques
    \item Principles and steps in penetrant testing
    \item Principles of magnetic particle testing
    \item Name advantages and disadvantages of NDT's
\end{enumerate}

\subsubsection{NDT Technique: Visual Inspection}
\begin{center}
    \centering
    \includegraphics[width=\textwidth]{Screenshot 2025-11-18 125439}

\end{center}
\subsubsection{NDT Technique: Penetrant testing (PT)}
Take advantage of capillary action of Penetrant
\begin{itemize}
    \item Good for surface discontinuities
    \item Can be used in production-type environment
    \item Flaws as fine as 1 micron can be detected
    \item Fluorescent dye can be used
    \item Portable
\end{itemize}
Advantages:
\begin{itemize}
    \item Inexpensive
    \item Sensitive
    \item Minimal equipment
    \item Can be applied to irregular shapes
    \item Versatile and Portable
    \item Minimal training required
    \item Ease of interpretation
\end{itemize}
Disadvantages:
\begin{itemize}
    \item 
\end{itemize}

\subsubsection{NDT Technique: Magnetic-Particle Inspection}





\subsection{Destructive testing}
Advantages:
\begin{itemize}
    \item Generates useful data for design purposes
    \item Data can be used to establish standards and specifications
    \item Data achieved through ?destructive? testing ususally quantitative.
    \item Service conditions can be measured
    \item Useful life can be predicted
\end{itemize}
Disadvantages:
\begin{itemize}
    \item specimens cant be used after testing
    \item expensive
\end{itemize}
\subsection{}
















\newpage
\section{Slide 1}
\begin{enumerate}
    \item Stages of environmental fractures
    \item Creep
    \item Third item
\end{enumerate}
\section{Slide 2 - Environmental Fracture}
slow or stable fracture due to combined action of loads and the environmentin a susceptible material
\subsection{example: Rolled up plastic hoses}
\section{Slide 3 - Stages of Environmental Fracture}
\begin{itemize}
    \item Stage 1: Accumulation of internal damage
    \item Stage 2: Nucleated crack turns into a macroscopic crack
    \item Stage 3: Rapid fracture, can happen $K<K_{IC}$
\end{itemize}
\section{Slide 4 - Creep}
\section{Stress corrosion cracking (SCC) Mechanism}
\begin{itemize}
    \item Slip-dissolution mechanism involves 4 stages: slip, film rupture, anodic dissolution, repassivation
    \item External stress causes dislocations to move and create slip steps, increasing plastic strain, and breaking the oxide film at the crak tip.
\end{itemize}
\section{SCC Mechanisms}
\subsection{Stress corrosion cracking SCC}
For stage I: $log(da/dt) = cK^m$\\
For stage II: $log(da/dt) = M/(zF\lambda)i_{curr}$
\subsection{SCC - Effect of Microstructure}
\subsection{Example problem}
$PLM = (500+273)[17 + log(28*365*24)]$
\\$PLM = 17190$\\
using the graph, $\sigma = 400 MPa$ 


\end{document}

