\documentclass[11pt]{article}
\usepackage[margin=1in]{geometry}
\usepackage{graphicx}
\usepackage{amsmath}
\usepackage{pgfplots}
\graphicspath{{../Screenshots/}}

\begin{document}
\section{Problem 1}
Data obtained from a series of stress rupture tests:
\begin{center}
    \includegraphics[width=\textwidth]{Screenshot 2025-11-24 161831.png}
\end{center}
\subsection{Make a Larson-Miller plot of the data}
\begin{center}
    \includegraphics[width=\textwidth]{LMParameter.png}
\end{center}
This plot is created using R. The Larson-Miller parameter is calculated using the equation:
$$\text{LMP} = T(20+\log(t))10^{-3}$$
For example, for the values at the top of the table associated with 650 $C^\circ$:
\\
$\text{LMP}=(650+273.15)(20+ \log 0.08)$
\subsection{Life span when 30 ksi of stress is applied at 600$C^\circ$}
\end{document}

