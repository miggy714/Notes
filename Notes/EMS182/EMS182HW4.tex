\documentclass[11pt]{article}
\usepackage[margin=1in]{geometry}
\usepackage{graphicx}
\usepackage{amsmath}
\usepackage{pgfplots}
\usepackage{titlesec}
\usepackage{url}
\usepackage{verbatim}
\graphicspath{{../Screenshots/}}
\newcommand{\memoheader}[4]{
  \noindent
  \textbf{From:} #2\\ 
  \textbf{To:} #1\\
  \textbf{Date:} #3\\
  \textbf{Subject:} #4\\[1ex]\hrule\vspace{1.5ex}
}
\titleformat{\section}
  {\normalfont\Large\bfseries}   % format
  {Problem~\thesection}          % label
  {1em}                          % sep
  {}                             % before-code

\renewcommand{\thesection}{\arabic{section}}





\begin{document}
\memoheader{Prof. Saedi}
            {Miguel Cuaycong}
            {\today}
            {EMS 182 - HW4}
\section{}
Data obtained from a series of stress rupture tests:
\begin{center}
    \includegraphics[width=\textwidth]{Screenshot 2025-11-24 161831.png}
\end{center}
\subsection{Make a Larson-Miller plot of the data}
\begin{center}
    \includegraphics[width=\textwidth]{LMP.png}
\end{center}
This plot is created using R. The Larson-Miller parameter is calculated using the equation:
$$\text{LMP} = T(20+\log(t))10^{-3}$$
For example, for the values at the top of the table associated with 650 $C^\circ$:
\\
$\sigma = 80$ ksi
\\
$\text{LMP}=(650+273.15)(20+ \log 0.08)10^{-3}=17.45$
\\~\\
The trendline is created using the highest degree that is statistically valid $(p\leq 0.05)$.
This recycles a code block I wrote for EME109 in R to fit experimental data points. Gen AI is used to improve plot aesthetics. 
Code-block is in the appendix with the prompt in the section labelled "Gen AI Code start"
\\~\\
Trendline equation: $y=0.531x^2-31.33x+467.92$
\subsection{Life span when 30 ksi of stress is applied at 600$C^\circ$}
<<<<<<< HEAD

\begin{align*}
    LMP
\end{align*}
=======
Using the trendline equation:
$t = \frac{31.33 \pm \sqrt{31.33^2-4(0.531)(437.92)}}{2(0.531)} = \mathbf{22.74}\text{ hours}$
\section{}
Effect of service temperature on useable stress levels for various metals
\subsection{Tungsten has a melting point of 3,400$C^\circ$. Why is it not considered for use in jet engines?}
\begin{center}
    \includegraphics[width = \textwidth]{Screenshot 2025-11-24 215213.png}
\end{center}
When looking up common material properties for tungsten(Tm), the following are the reasons why its not considered for jet engines:
\begin{itemize}
    \item Low ductility
    \item High density ($19.3 g/cm^3$)
    \item Tungsten oxides that develop as a protective film melts at a lower temperature than Tm
\end{itemize}
\subsection{Advantages of aluminum alloys over refractory materials at operating temperatures of 400$F^\circ$}
As seen in the graph above, Aluminum alloys have a bigger range and a higher allowable stress compared to refractory materials at that range.
\section{}
Creep mechanisms:
\begin{enumerate}
    \item Dislocation-controlled
    \item Diffusion-controlled
    \begin{enumerate}
        \item Grain boundary Diffusion
        \item Lattice Diffusion
    \end{enumerate}
\end{enumerate}
For Dislocation creep, strain rate is determined by vacancy motion. It is dependent on the stress and independent of grain size. 
For Lattice diffusion, grain size decreases strain as well as temperature. 
For grain boundary diffusion, grain size is cubed which means it more sensitive to grain size.
\\~\\
\begin{tabular}{|c|c|c|}
    \hline
    - & low T & high T
    \\
    \hline
    high $\sigma$ & Dislocation controlled & Dislocation controlled
    \\
    \hline
    low $\sigma$ & GB diffusion & Lattice diffusion
    \\
    \hline

\end{tabular}
\section{compare abrasive wear and adhesion wear mechanisms}
Abrasive wear occurs when two materials pressed against one another have different hardness. 
Plowing, cutting, and/or fracture can occur. Wear rate decreases as fracture toughness increases. Adhesion wear occurs when two metals are soluble to each other.
When in contact, the two metals bond and if that adhesive bond is stronger than the cohesion within a metal, wear occurs.
\section{}
During the cold rolling of stainless steel, tribological interactions between the work rolls and the strip surface can lead to surface defects such as scratches.
\subsection{What operational or material factors can be adjusted to minimize these defects, and how does each factor contribute to their reduction?}
The use of proper lubrication during the cold rolling process can maintain separation of the surface and reduce heat generated. 
Making the roller be less than or at the same surface toughness as the steel will lessen the wear on the manufactured material. 
Properly isolated and sanitized processes to avoid three body abrasion will reduce abrasive wear.
\subsection{Which factors are inherent to the process or materials and therefore cannot be easily changed?}
Cold rolling will alwyas be a high stress process which will strain hardening to the steel changing the surface properties to be harder and more brittle.
\section{}
Extra credit: Explain how microstructural factors such as grain size, dislocation density (cold work), and inclusion content can affect SCC. (3 points)
\\~\\
External stress applied to a metal will produce deeper dislocations the higher the grain size. This can help SCC reach deeper depths. Dislocation density makes crack initiation easier which sets the stage for SCC.
Inclusion content can function like HIC, inclusions can oxidize or have other chemical reactions to the environment which can affect SCC.
\newpage
\appendix
\section*{Appendix: R-code for Stress--LMP plot}
\begin{verbatim}
tempC  <- c(650,650,650,650,
            730,730,730,730,
            815,815,815,
            920,920,920,920,
            1040,1040,1040)

stress <- c(80,65,60,40,
            60,50,30,25,
            50,30,20,
            30,20,15,10,
            20,10,5)

rlife  <- c(0.08,8.5,28,483,
            0.2,1.8,127,1023,
            0.3,3.1,332,
            0.08,1.3,71,123,
            0.3,1,28)

d <- data.frame(tempC=tempC, 
                stress_ksi=stress, 
                rlife_hr=rlife)

# Compute LMP = 1e-3 * T(K) * (20 + log10(t_hr))
d$T_K  <- d$tempC + 273.15 #C to K
d$LMP  <- 1e-3 * d$T_K * (20 + log10(d$rlife_hr))

head(d)


#Simple algorithm to determine highest order statistically valid polynomial fit
bestfit <- function(f, x, p=0.05, maxdeg = 4) {
  for (d in 1:maxdeg) {
    fit_d <- lm(f ~ poly(x, d, raw = TRUE)) # builds x1, x1^2, ..., x1^d
    p_high <- summary(fit_d)$coefficients[d + 1, 4]  # p of x1^d
    if (p_high <= p) {
      fit_best <- fit_d
      best_deg <- d
    } else {
      break
    }
  }
  fit_best
}

Ffit <- bestfit(d$stress_ksi, d$LMP)

# Evaluate an lm fit made with poly(x, d, raw=TRUE) at numeric vector xnew
eval_poly_fit <- function(fit, xnew) {
  b <- coef(fit)             # c(b0, b1, ..., bd)
  d <- length(b) - 1
  X <- cbind(1, sapply(1:d, function(p) xnew^p))  # columns: 1, x, x^2, ..., x^d
  as.vector(X %*% b)
}

LMP_grid <- seq(min(d$LMP), max(d$LMP), length.out = 300) # x axis
pred_stress <- eval_poly_fit(Ffit, LMP_grid) # trendline

## Gen AI code start-- PROMPT: Use the code above as a base to construct a plot with 
## datapoints and the trendline

# Base plot: data points
plot(
  d$LMP, d$stress_ksi,
  pch = 19, col = "steelblue",
  xlab = "LMP (1e-3 * K * (20 + log10(t_hr)))",
  ylab = "Stress (ksi)",
  main = "Stress vs LMP with Polynomial Fit"
)

# Add fitted curve
lines(LMP_grid, pred_stress, col = "firebrick", lwd = 2)

# Optional: annotate polynomial degree and R^2
deg <- length(coef(Ffit)) - 1
R2 <- summary(Ffit)$r.squared
legend(
  "topright",
  legend = c("Data", sprintf("Fit: degree %d (R^2 = %.3f)", deg, R2)),
  col = c("steelblue", "firebrick"),
  pch = c(19, NA),
  lty = c(NA, 1),
  lwd = c(NA, 2),
  bty = "n"
) 
\end{verbatim}
>>>>>>> ef6786fca68962b1d8a483cf83ba7c67baed8efd
\end{document}

