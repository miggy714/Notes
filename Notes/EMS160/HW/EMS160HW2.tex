\documentclass[11pt]{article}
\usepackage{graphicx}
\usepackage{float}
\usepackage{amsmath}
\usepackage{enumerate}
\usepackage{pgfplots}
\title{UC Davis
\\~\\
EMS 160 Homework 2
\\~\\
Enthalpy, Entropy and Gibbs
\\~\\
Prof. Scott McCormack}

\author{Xiaokun Yang\\Miguel Cuaycong\\ Jonathan Mases}
\date{\today}

\begin{document}
\maketitle
\newpage
\section*{Question 4 (6pts)}
\begin{enumerate}[i]
    \item Rankine cycle
    \\
    \begin{tikzpicture}
    \begin{axis}[
        title={Rankine cycle},
        xlabel={Entropy},
        yticklabel=\empty,
        xticklabel=\empty,
    ]

    \end{axis}
    \end{tikzpicture}
    \\
    1-2: Isothermal heat addition: $Q = \int_{1}^{2} T dS, W = 0, Q \text{ is positive}$ 
    \\
    2-3: Isentropic expansion: $Q = 0, W = V(P_3 - P_2), W \text{ is negative}$
    \\
    3-4: Isothermal heat rejection: $Q = \int_{3}^{4} T dS, W = 0,Q \text{ is negative} $
    \\
    4-1: Isentropic compression: $Q = 0,  W = V(P_1-P_4), W \text{ is positive}$
    \\~\\
    Commonly used in power plants.
    \newpage
    \item Stirling cycle
        \\
    \begin{tikzpicture}
    \begin{axis}[
        title={Stirling cycle},
        xlabel={Entropy},
        yticklabel=\empty,
        xticklabel=\empty,
    ]
    \end{axis}
    \end{tikzpicture}
    \\
    1-2: Isothermal heat addition: $Q = \int_{1}^{2} T dS$
    \\
    2-3: Isochoric heat rejection: $Q = \int_{2}^{3} T dS$
    \\
    3-4: Isothermal heat rejection: $Q = \int_{3}^{4} T dS$
    \\
    4-1: Isochoric heat addition: $Q = \int_{4}^{1} T dS$
    \\
    net work in this cycle: $W_{net} = mR\ln{(\frac{V_2}{V_1})}\Delta T$
    \\~\\
     Invented in the early 1800. Aside from use in engines, it is also used as coolers
    \newpage
     \item Otto cycle
        \\
    \begin{tikzpicture}
    \begin{axis}[
        title={Otto cycle},
        xlabel={Entropy},
        yticklabel=\empty,
        xticklabel=\empty,
    ]
    \end{axis}
    \end{tikzpicture}
    \\
    Stroke 1
    \\
    1-2: Isentropic and Adiabatic compression: $Q = 0, W = U_2 - U_1$
    \\
    Stroke 2
    \\
    2-3: Isochoric heat addition: $Q = \int_{2}^{3} T dS$
    \\
    Stroke 3
    \\
    3-4: Isentropic and Adiabatic expansion: $Q = 0, W = U_4 - U_3$
    \\
    Stroke 4
    \\
    4-1: Isochoric heat rejection: $Q = \int_{4}^{1} T dS$
    \\~\\
    Used in most piston driven engines in automobiles
    \\
    \newpage
    \item Brayton cycle
        \\
    \begin{tikzpicture}
    \begin{axis}[
        title={Brayton cycle},
        xlabel={Entropy},
        yticklabel=\empty,
        xticklabel=\empty,
    ]
    \end{axis}
    \end{tikzpicture} 
    \\
    1-2: Isentropic compression: $Q = 0, W = H_2-H_1$
    \\
    2-3: Isobaric heat addition: $Q = \int_{2}^{3} T dS, W = 0$
    \\
    3-4: Isentropic expansion: $Q = 0, W = H_2-H_1$
    \\
    4-1: Isobaric heat addition: $Q = \int_{4}^{1} T dS, W = 0$
    \\~\\
    Applications in power grids as gas turbines. Also used in aircraft jet engines
\end{enumerate} 




\end{document}
